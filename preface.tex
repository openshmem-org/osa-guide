\section*{Preface}
\addcontentsline{toc}{section}{Preface}

This User's Manual describes the \openshmem
Analyzer~\cite{hernandez2012openshmem}, a tool that provides source
code analysis and correctness checks capabilities to the user for
\openshmem programs. It provides a range of information about the
source program in textual or graphical format. The \openshmem Analyzer
relies on web browsers and the graph library GraphViz to render its
graphs and use of hyperlinks to navigate to the source code and
warning messages.

\subsection*{Audience Description}
\addcontentsline{toc}{subsection}{Audience Description}

This manual is a user's guide only. Users are expected to have a basic
knowledge of the structure of programs and some experience with C and
\openshmem programming. The \openshmem Analyzer is available on most
GNU/Linux distributions (e.g.\ Red Hat and SUSE) with x86/x86-64
processors. It is assumed that users are familiar with the basic
commands on these systems.

\subsection*{Organization}
\addcontentsline{toc}{subsection}{Organization}

This User's Guide is structured into the following sections and
appendices:

\begin{description}
\item[Section~\ref{chapter:introduction}:~\nameref{chapter:introduction}]
  gives an overview of the \openshmem Analyzer Tool and describes its
  major goals;
\item[Section~\ref{chapter:basic-usage}:~\nameref{chapter:basic-usage}]
  describes naming conventions, generated files and how to visualize
  the graphs and traverse the source code;
\item[Section~\ref{chapter:features}:~\nameref{chapter:features}]
  describes the features for local and global analysis of \openshmem
  programs and how to interpret the information within the graphs;
\item [Appendix~\ref{chapter:download}:~\nameref{chapter:download}]
  describes how to obtain and install the \openshmem Analyzer either
  as a pre-built executable or in source form;
\item [Appendix~\ref{chapter:future}:~\nameref{chapter:future}] is a
  quick view of on-going work of the \openshmem Analyzer and its
  future functionality.
\end{description}

\subsection*{Status of Work}
\addcontentsline{toc}{subsection}{Status of Work}

The tool is currently in its infancy and there are many enhancements
to be made, both with analysis capabilities and with its use and
invocation.
