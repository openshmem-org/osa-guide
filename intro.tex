\section{Introduction}
\label{chapter:introduction}

In this chapter we give an overview of the overall features of the
\openshmem Analyzer and describe the structure of the system.

\subsection{Major Goals of the \openshmem Analyzer}

Whenever application software is developed or ported to \openshmem
from a new or an existing one, the source code must be carefully
analyzed in order to understand many details of the current
implementation while at the same time avoid common errors in
\openshmem applications. The \openshmem Analyzer is a tool to support
an application developer or code owner who wishes to understand their
C application better. It provides a range of information including the
structure of a source program in a graphical browseable form. The
current input languages for the \openshmem Analyzer are C/C++,
\openshmem API 1.0.

\openshmem is a standard for SHMEM library implementations. Many SHMEM
libraries exist but they do not conform to a particular standard and
have similar but not identical APIs and behavior, which hinders
portability. However, significant user efforts are required to
parallelize serial codes with \openshmem or further analyze and
optimize the performance of \openshmem programs. The \openshmem
Analyzer, with its comprehensive intra and inter procedural analysis
information, can be an indispensable assistant for writing, analyzing
and optimizing \openshmem applications.  The \openshmem Analyzer has
an interactive component that provides a graphical user interface
using HTML to display its output and to navigate the source code and
error messages within them.

The \openshmem Analyzer is an on-going research project developed at
Oak Ridge National Laboratories with funding from DOD. Its
functionality is based on the OpenUH compiler infrastructure, which is
maintained by the HPCTools Group at the University of
Houston.

The \openshmem Analyzer is intended to be an open source tool
available for the \openshmem community.
