\section{Ongoing and Future Work}
\label{chapter:future}

We plan to develop a stand-alone user interface to present an
interactive callgraph and control flow graph that are aware of
\openshmem calls and develop a triage between these different views.
The idea is that there will be a data flow view that originates from
\openshmem calls and enables the user to see the use-define chains that
can help to keep track of the use and definition of pointers in
\openshmem calls. This will help the user evaluate the placement of a
particular call and how it is related to the rest of the application
more intuitively.  In addition, we could develop a view to see how
symmetric variables are accessed in the entire application, since
symmetric variables can affect many procedures (e.g.\ global variables,
interprocedural pointers). These sorts of views are important to
understand the overall side effects of the application.

The \openshmem Analyzer is moving toward a proper infrastructure to
perform parallel data flow analysis is needed, and apply the
state-of-the-art and how this relates to the context of \openshmem. As
a first step, we will explore the concept of program slicing at the
process level, which is the process for separating the different
control flow graphs from different processes. The idea is to simplify
the control flow graphs of a program per process and correctly denote
synchronizations across them. Through graph analysis we can classify
the different process control-flowgraphs into subsets that can be used
to represent optimizations and where classical program optimizations
can be applied per \openshmem task.

We plan to explore how to integrate better the \openshmem Analyzer to
the \openshmem library in a way that is more compiler friendly. We
need to make sure the library implementation allows the compiler to
analyze and optimize it together with the application. This will
include the inlining of all \openshmem calls, their specialization to
the application context, constant propagation/dead code elimination,
and elimination of redundant runtime checks. The \openshmem Analyzer
could perform checks at compile time and define a set of assertions
that we can enforce at runtime to make sure the library is run in the
right context, reducing runtime checks and
overheads~\cite{SwaroopOpenSHMEM14,SwaroopPGAS13}.

We will also explore how to integrate dynamic information to the
\openshmem Analyzer. This will mean integrating it with performance
tools such as TAU and the \openshmem Tracer. We will combine the
\openshmem Analyzer instrumentation with the \openshmem wrappers of
these tools to gather calculate frequently executed paths in the
control graph and callgraph or values of variables at a given point of
execution. This will help toward feedback and present to the user a
dynamic callgraph and control flow graph, that shows the execution
code coverage and the frequently accessed code path that can be
specialized.
