\section{Download and Install the \openshmem Analyzer}
\label{chapter:download}

\subsection{Download Location}

The project website for the \openshmem Analyzer is:

\begin{center}
\url{http://www.openshmem.org/OSA}
\end{center}

\subsection{Installation}

The \openshmem Analyzer is available in the following ways:

\subsubsection{Pre-built Executable}

For immediate use, a tarball of the \openshmem Analyzer, called
\begin{center}
\texttt{openuh-3.0.38-x86\_64-bin.tar.bz2}
\footnote{3.0.38 is the version at time of writing but will change in
  the future.}
\end{center}
can be downloaded via the project website above.  Then

\begin{enumerate}
\item extract the contents of the tarball to a directory, call it
  \texttt{prefix};
\item prepend the directory \texttt{prefix/openuh-3.0.38/bin} to your
  \texttt{PATH} environment variable.
\end{enumerate}

\subsubsection{Source}

The full source code of the OpenUH compiler containing the \openshmem
Analyzer can also be downloaded from a repository via the project
website above.

Configuration is via the common GNU Autotools \texttt{configure}
command, so you can build \emph{in situ} or in a separate build
directory.  The \openshmem Analyzer will be built when the
\texttt{-{}-{}enable-osa} configure flag is used.  Then do the usual
\texttt{make}/\texttt{make install} sequence.

\subsubsection{Tool Infrastructure Documentation}

A paper describing the OpenUH compiler that is the infrastructure of
the \openshmem Analyzer is in~\cite{chapman2012experiencesspringer}.
